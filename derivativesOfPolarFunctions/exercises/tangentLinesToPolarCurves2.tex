\documentclass{ximera}

%\usepackage{todonotes}

\newcommand{\todo}{}

\usepackage{esint} % for \oiint
\ifxake%%https://math.meta.stackexchange.com/questions/9973/how-do-you-render-a-closed-surface-double-integral
\renewcommand{\oiint}{{\large\bigcirc}\kern-1.56em\iint}
\fi


\graphicspath{
  {./}
  {ximeraTutorial/}
  {basicPhilosophy/}
  {functionsOfSeveralVariables/}
  {normalVectors/}
  {lagrangeMultipliers/}
  {vectorFields/}
  {greensTheorem/}
  {shapeOfThingsToCome/}
  {dotProducts/}
  {partialDerivativesAndTheGradientVector/}
  {../productAndQuotientRules/exercises/}
  {../normalVectors/exercisesParametricPlots/}
  {../continuityOfFunctionsOfSeveralVariables/exercises/}
  {../partialDerivativesAndTheGradientVector/exercises/}
  {../directionalDerivativeAndChainRule/exercises/}
  {../commonCoordinates/exercisesCylindricalCoordinates/}
  {../commonCoordinates/exercisesSphericalCoordinates/}
  {../greensTheorem/exercisesCurlAndLineIntegrals/}
  {../greensTheorem/exercisesDivergenceAndLineIntegrals/}
  {../shapeOfThingsToCome/exercisesDivergenceTheorem/}
  {../greensTheorem/}
  {../shapeOfThingsToCome/}
  {../separableDifferentialEquations/exercises/}
}

\newcommand{\mooculus}{\textsf{\textbf{MOOC}\textnormal{\textsf{ULUS}}}}

\usepackage{tkz-euclide}\usepackage{tikz}
\usepackage{tikz-cd}
\usetikzlibrary{arrows}
\tikzset{>=stealth,commutative diagrams/.cd,
  arrow style=tikz,diagrams={>=stealth}} %% cool arrow head
\tikzset{shorten <>/.style={ shorten >=#1, shorten <=#1 } } %% allows shorter vectors

\usetikzlibrary{backgrounds} %% for boxes around graphs
\usetikzlibrary{shapes,positioning}  %% Clouds and stars
\usetikzlibrary{matrix} %% for matrix
\usepgfplotslibrary{polar} %% for polar plots
\usepgfplotslibrary{fillbetween} %% to shade area between curves in TikZ
\usetkzobj{all}
\usepackage[makeroom]{cancel} %% for strike outs
%\usepackage{mathtools} %% for pretty underbrace % Breaks Ximera
%\usepackage{multicol}
\usepackage{pgffor} %% required for integral for loops



%% http://tex.stackexchange.com/questions/66490/drawing-a-tikz-arc-specifying-the-center
%% Draws beach ball
\tikzset{pics/carc/.style args={#1:#2:#3}{code={\draw[pic actions] (#1:#3) arc(#1:#2:#3);}}}



\usepackage{array}
\setlength{\extrarowheight}{+.1cm}
\newdimen\digitwidth
\settowidth\digitwidth{9}
\def\divrule#1#2{
\noalign{\moveright#1\digitwidth
\vbox{\hrule width#2\digitwidth}}}





\newcommand{\RR}{\mathbb R}
\newcommand{\R}{\mathbb R}
\newcommand{\N}{\mathbb N}
\newcommand{\Z}{\mathbb Z}

\newcommand{\sagemath}{\textsf{SageMath}}


%\renewcommand{\d}{\,d\!}
\renewcommand{\d}{\mathop{}\!d}
\newcommand{\dd}[2][]{\frac{\d #1}{\d #2}}
\newcommand{\pp}[2][]{\frac{\partial #1}{\partial #2}}
\renewcommand{\l}{\ell}
\newcommand{\ddx}{\frac{d}{\d x}}

\newcommand{\zeroOverZero}{\ensuremath{\boldsymbol{\tfrac{0}{0}}}}
\newcommand{\inftyOverInfty}{\ensuremath{\boldsymbol{\tfrac{\infty}{\infty}}}}
\newcommand{\zeroOverInfty}{\ensuremath{\boldsymbol{\tfrac{0}{\infty}}}}
\newcommand{\zeroTimesInfty}{\ensuremath{\small\boldsymbol{0\cdot \infty}}}
\newcommand{\inftyMinusInfty}{\ensuremath{\small\boldsymbol{\infty - \infty}}}
\newcommand{\oneToInfty}{\ensuremath{\boldsymbol{1^\infty}}}
\newcommand{\zeroToZero}{\ensuremath{\boldsymbol{0^0}}}
\newcommand{\inftyToZero}{\ensuremath{\boldsymbol{\infty^0}}}



\newcommand{\numOverZero}{\ensuremath{\boldsymbol{\tfrac{\#}{0}}}}
\newcommand{\dfn}{\textbf}
%\newcommand{\unit}{\,\mathrm}
\newcommand{\unit}{\mathop{}\!\mathrm}
\newcommand{\eval}[1]{\bigg[ #1 \bigg]}
\newcommand{\seq}[1]{\left( #1 \right)}
\renewcommand{\epsilon}{\varepsilon}
\renewcommand{\phi}{\varphi}


\renewcommand{\iff}{\Leftrightarrow}

\DeclareMathOperator{\arccot}{arccot}
\DeclareMathOperator{\arcsec}{arcsec}
\DeclareMathOperator{\arccsc}{arccsc}
\DeclareMathOperator{\si}{Si}
\DeclareMathOperator{\scal}{scal}
\DeclareMathOperator{\sign}{sign}


%% \newcommand{\tightoverset}[2]{% for arrow vec
%%   \mathop{#2}\limits^{\vbox to -.5ex{\kern-0.75ex\hbox{$#1$}\vss}}}
\newcommand{\arrowvec}[1]{{\overset{\rightharpoonup}{#1}}}
%\renewcommand{\vec}[1]{\arrowvec{\mathbf{#1}}}
\renewcommand{\vec}[1]{{\overset{\boldsymbol{\rightharpoonup}}{\mathbf{#1}}}}
\DeclareMathOperator{\proj}{\mathbf{proj}}
\newcommand{\veci}{{\boldsymbol{\hat{\imath}}}}
\newcommand{\vecj}{{\boldsymbol{\hat{\jmath}}}}
\newcommand{\veck}{{\boldsymbol{\hat{k}}}}
\newcommand{\vecl}{\vec{\boldsymbol{\l}}}
\newcommand{\uvec}[1]{\mathbf{\hat{#1}}}
\newcommand{\utan}{\mathbf{\hat{t}}}
\newcommand{\unormal}{\mathbf{\hat{n}}}
\newcommand{\ubinormal}{\mathbf{\hat{b}}}

\newcommand{\dotp}{\bullet}
\newcommand{\cross}{\boldsymbol\times}
\newcommand{\grad}{\boldsymbol\nabla}
\newcommand{\divergence}{\grad\dotp}
\newcommand{\curl}{\grad\cross}
%\DeclareMathOperator{\divergence}{divergence}
%\DeclareMathOperator{\curl}[1]{\grad\cross #1}
\newcommand{\lto}{\mathop{\longrightarrow\,}\limits}

\renewcommand{\bar}{\overline}

\colorlet{textColor}{black}
\colorlet{background}{white}
\colorlet{penColor}{blue!50!black} % Color of a curve in a plot
\colorlet{penColor2}{red!50!black}% Color of a curve in a plot
\colorlet{penColor3}{red!50!blue} % Color of a curve in a plot
\colorlet{penColor4}{green!50!black} % Color of a curve in a plot
\colorlet{penColor5}{orange!80!black} % Color of a curve in a plot
\colorlet{penColor6}{yellow!70!black} % Color of a curve in a plot
\colorlet{fill1}{penColor!20} % Color of fill in a plot
\colorlet{fill2}{penColor2!20} % Color of fill in a plot
\colorlet{fillp}{fill1} % Color of positive area
\colorlet{filln}{penColor2!20} % Color of negative area
\colorlet{fill3}{penColor3!20} % Fill
\colorlet{fill4}{penColor4!20} % Fill
\colorlet{fill5}{penColor5!20} % Fill
\colorlet{gridColor}{gray!50} % Color of grid in a plot

\newcommand{\surfaceColor}{violet}
\newcommand{\surfaceColorTwo}{redyellow}
\newcommand{\sliceColor}{greenyellow}




\pgfmathdeclarefunction{gauss}{2}{% gives gaussian
  \pgfmathparse{1/(#2*sqrt(2*pi))*exp(-((x-#1)^2)/(2*#2^2))}%
}


%%%%%%%%%%%%%
%% Vectors
%%%%%%%%%%%%%

%% Simple horiz vectors
\renewcommand{\vector}[1]{\left\langle #1\right\rangle}


%% %% Complex Horiz Vectors with angle brackets
%% \makeatletter
%% \renewcommand{\vector}[2][ , ]{\left\langle%
%%   \def\nextitem{\def\nextitem{#1}}%
%%   \@for \el:=#2\do{\nextitem\el}\right\rangle%
%% }
%% \makeatother

%% %% Vertical Vectors
%% \def\vector#1{\begin{bmatrix}\vecListA#1,,\end{bmatrix}}
%% \def\vecListA#1,{\if,#1,\else #1\cr \expandafter \vecListA \fi}

%%%%%%%%%%%%%
%% End of vectors
%%%%%%%%%%%%%

%\newcommand{\fullwidth}{}
%\newcommand{\normalwidth}{}



%% makes a snazzy t-chart for evaluating functions
%\newenvironment{tchart}{\rowcolors{2}{}{background!90!textColor}\array}{\endarray}

%%This is to help with formatting on future title pages.
\newenvironment{sectionOutcomes}{}{}



%% Flowchart stuff
%\tikzstyle{startstop} = [rectangle, rounded corners, minimum width=3cm, minimum height=1cm,text centered, draw=black]
%\tikzstyle{question} = [rectangle, minimum width=3cm, minimum height=1cm, text centered, draw=black]
%\tikzstyle{decision} = [trapezium, trapezium left angle=70, trapezium right angle=110, minimum width=3cm, minimum height=1cm, text centered, draw=black]
%\tikzstyle{question} = [rectangle, rounded corners, minimum width=3cm, minimum height=1cm,text centered, draw=black]
%\tikzstyle{process} = [rectangle, minimum width=3cm, minimum height=1cm, text centered, draw=black]
%\tikzstyle{decision} = [trapezium, trapezium left angle=70, trapezium right angle=110, minimum width=3cm, minimum height=1cm, text centered, draw=black]


\author{Jason Miller}
\license{Creative Commons 3.0 By-bC}


\outcome{}

\begin{document}
\begin{exercise}


Consider the polar curve $r=\sin(2\theta)$. The graph is show below. 




\begin{image}  
  \begin{tikzpicture}  
    \begin{axis}[  
        xmin=-1.5,  
        xmax=1.5,  
        ymin=-1.5,  
        ymax=1.5,  
        axis lines=center,  
        xlabel=$x$,  
        ylabel=$y$,  
        every axis y label/.style={at=(current axis.above origin),anchor=south},  
        every axis x label/.style={at=(current axis.right of origin),anchor=west},  
      ]  
      \addplot[data cs=polar,penColor,domain=0:360,samples=360,smooth, thick] (x,{sin(2*x)});
      \end{axis}  
  \end{tikzpicture}  
\end{image} 

We want to find tangent line to the curve when $\theta=\frac{\pi}{6}$. 

First find the point ( in Cartesian coordinates) on the curve $r=\sin(2\theta)$ when $\theta=\frac{\pi}{6}$. 
The point is $\left( \answer{ \frac{3}{4}},  \answer{ \frac{\sqrt{3}}{4}} \right)$. 

\begin{hint}

Recall that given the polar coordinates $(r,\theta)$ of a point, we can find the Cartesian coordinates $(x, y)$, using
\begin{align*}
x&=r\cos(\theta) \\
y&=r\sin(\theta)
\end{align*}

Since we want a point on the curve $r=1+\sin(\theta)$, we can substitute for $r$ and get:

\begin{align*}
x&=\answer{\sin(2\theta)\cos(\theta)} \\
y&=\answer{\sin(2\theta)\sin(\theta) }
\end{align*}

Now since we want the point corresponding to $\theta=\frac{\pi}{6}$ we have:


\begin{align*}
x&=\answer{ \frac{3}{4} } \\
y&=\answer{\frac{\sqrt{3}}{4}  }
\end{align*}

\end{hint}

\begin{exercise}

Find the slope of tangent line to the curve when $\theta=\frac{\pi}{6}$. 

The slope is $\answer{ \frac{5}{\sqrt{3}} }$.

\begin{hint}

Consider the formulas for changing from polar coordinates to Cartesian coordinates:

\begin{align*}
x&=r\cos(\theta) \\
y&=r\sin(\theta)
\end{align*}

Using the equation of our curve $r=1+\sin(\theta)$, we can substitute for $r$ to obtain:

\begin{align*}
x&=\answer{\sin(2\theta)\cos(\theta)} \\
y&=\answer{\sin(2\theta)\sin(\theta) }
\end{align*}

Note that we have expressed both the $x$ and $y$ coordinates of the points of the curve in terms of functions of a single parameter $\theta$. That means we have a parametric description for our curve in terms of $\theta$. 

Recall that for parametric equations: 

\[
\begin{cases}
x&=x(t) \\
y&=y(t)
\end{cases}
\]

we find the slope of the tangent line at a point corresponding to the parameter value $t$ by using: 

\[
\dd[y]{x}=\frac{ y'(t)}{x'(t)}
\]

Since we have a parametric description of our curve in terms of $\theta$, 

\begin{align*}
x&=\answer{\sin(2\theta)\cos(\theta)} \\
y&=\answer{\sin(2\theta)\sin(\theta) }
\end{align*}

we use the same method for finding the slope. 

So we need to calculate:

$\dd[y]{x}=\frac{ y'(\theta)}{x'(\theta)}$ and evaluate it at $\theta=\frac{\pi}{6}$. 

Calculating these derivatives, we obtain:

\begin{align*}
x'(\theta)&=\answer{ 2\cos(2\theta)\cos(\theta)- \sin(2\theta)\sin(\theta)} \\
y'(\theta)&=\answer{ 2\cos(2\theta)\sin(\theta)+\sin(2\theta)\cos(\theta) }
\end{align*} 

Therefore $\dd[y]{x}$ evaluated when $\theta=\frac{\pi}{6}$ is $\answer{ \frac{5}{\sqrt{3}}  }$.


\end{hint}

\begin{exercise}

The tangent line to the curve when $\theta=\frac{\pi}{6}$ is given by: 

$y-\answer{\frac{\sqrt{3}}{4}   }=\answer{ \frac{5}{\sqrt{3}}  }\left( x- \answer{  \frac{3}{4} }  \right)$. 

\begin{hint}


The point-slope formula for a line is give by $y-y_0=m (x-x_0)$ where $m$ is the slope of the line and $(x_{0},y_{0})$ is a given point on the line. 
Since we want the tangent line to the curve $r=\sin(2\theta)$ when $\theta=\frac{\pi}{6}$, we need $(x_{0},y_{0})$ to be the point on the curve corresponding to the parameter value $\theta=\frac{\pi}{6}$ (we found this point earlier). The slope $m$ will be $\dd[y]{x}$ evaluated when $\theta=\frac{\pi}{6}$, which we also have previously calculated. 

\end{hint}

\begin{exercise}
Plot both the original curve and the tangent line in Desmos to verify that we have found the correct tangent line.  

From the graph, it looks like the tangent line intersects the curve in $\answer{3}$ places (including the point of tangency).


\begin{hint}
You should see the following picture:
The tangent line is shown below in red:

\begin{image}  
  \begin{tikzpicture}  
    \begin{axis}[  
        xmin=-1.5,  
        xmax=1.5,  
        ymin=-1.5,  
        ymax=1.5,  
        axis lines=center,  
        xlabel=$x$,  
        ylabel=$y$,  
        every axis y label/.style={at=(current axis.above origin),anchor=south},  
        every axis x label/.style={at=(current axis.right of origin),anchor=west},  
      ]  
      \addplot[data cs=polar,penColor,domain=0:360,samples=360,smooth, thick] (x,{sin(2*x)});
      \draw (axis cs:2,2.9) node { $r=\sin 2\theta$};
      \addplot[ penColor2, domain=0:2, thick] {.433+2.887*(x-.75)};
     \addplot[only marks, mark=*] coordinates {(.75, .433)};
         \end{axis}  
  \end{tikzpicture}  
\end{image} 


\end{hint}

\end{exercise}
\end{exercise}
\end{exercise}
\end{exercise}
\end{document}
