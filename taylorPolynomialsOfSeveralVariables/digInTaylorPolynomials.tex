\documentclass{ximera}

%\usepackage{todonotes}

\newcommand{\todo}{}

\usepackage{esint} % for \oiint
\ifxake%%https://math.meta.stackexchange.com/questions/9973/how-do-you-render-a-closed-surface-double-integral
\renewcommand{\oiint}{{\large\bigcirc}\kern-1.56em\iint}
\fi


\graphicspath{
  {./}
  {ximeraTutorial/}
  {basicPhilosophy/}
  {functionsOfSeveralVariables/}
  {normalVectors/}
  {lagrangeMultipliers/}
  {vectorFields/}
  {greensTheorem/}
  {shapeOfThingsToCome/}
  {dotProducts/}
  {partialDerivativesAndTheGradientVector/}
  {../productAndQuotientRules/exercises/}
  {../normalVectors/exercisesParametricPlots/}
  {../continuityOfFunctionsOfSeveralVariables/exercises/}
  {../partialDerivativesAndTheGradientVector/exercises/}
  {../directionalDerivativeAndChainRule/exercises/}
  {../commonCoordinates/exercisesCylindricalCoordinates/}
  {../commonCoordinates/exercisesSphericalCoordinates/}
  {../greensTheorem/exercisesCurlAndLineIntegrals/}
  {../greensTheorem/exercisesDivergenceAndLineIntegrals/}
  {../shapeOfThingsToCome/exercisesDivergenceTheorem/}
  {../greensTheorem/}
  {../shapeOfThingsToCome/}
  {../separableDifferentialEquations/exercises/}
}

\newcommand{\mooculus}{\textsf{\textbf{MOOC}\textnormal{\textsf{ULUS}}}}

\usepackage{tkz-euclide}\usepackage{tikz}
\usepackage{tikz-cd}
\usetikzlibrary{arrows}
\tikzset{>=stealth,commutative diagrams/.cd,
  arrow style=tikz,diagrams={>=stealth}} %% cool arrow head
\tikzset{shorten <>/.style={ shorten >=#1, shorten <=#1 } } %% allows shorter vectors

\usetikzlibrary{backgrounds} %% for boxes around graphs
\usetikzlibrary{shapes,positioning}  %% Clouds and stars
\usetikzlibrary{matrix} %% for matrix
\usepgfplotslibrary{polar} %% for polar plots
\usepgfplotslibrary{fillbetween} %% to shade area between curves in TikZ
\usetkzobj{all}
\usepackage[makeroom]{cancel} %% for strike outs
%\usepackage{mathtools} %% for pretty underbrace % Breaks Ximera
%\usepackage{multicol}
\usepackage{pgffor} %% required for integral for loops



%% http://tex.stackexchange.com/questions/66490/drawing-a-tikz-arc-specifying-the-center
%% Draws beach ball
\tikzset{pics/carc/.style args={#1:#2:#3}{code={\draw[pic actions] (#1:#3) arc(#1:#2:#3);}}}



\usepackage{array}
\setlength{\extrarowheight}{+.1cm}
\newdimen\digitwidth
\settowidth\digitwidth{9}
\def\divrule#1#2{
\noalign{\moveright#1\digitwidth
\vbox{\hrule width#2\digitwidth}}}





\newcommand{\RR}{\mathbb R}
\newcommand{\R}{\mathbb R}
\newcommand{\N}{\mathbb N}
\newcommand{\Z}{\mathbb Z}

\newcommand{\sagemath}{\textsf{SageMath}}


%\renewcommand{\d}{\,d\!}
\renewcommand{\d}{\mathop{}\!d}
\newcommand{\dd}[2][]{\frac{\d #1}{\d #2}}
\newcommand{\pp}[2][]{\frac{\partial #1}{\partial #2}}
\renewcommand{\l}{\ell}
\newcommand{\ddx}{\frac{d}{\d x}}

\newcommand{\zeroOverZero}{\ensuremath{\boldsymbol{\tfrac{0}{0}}}}
\newcommand{\inftyOverInfty}{\ensuremath{\boldsymbol{\tfrac{\infty}{\infty}}}}
\newcommand{\zeroOverInfty}{\ensuremath{\boldsymbol{\tfrac{0}{\infty}}}}
\newcommand{\zeroTimesInfty}{\ensuremath{\small\boldsymbol{0\cdot \infty}}}
\newcommand{\inftyMinusInfty}{\ensuremath{\small\boldsymbol{\infty - \infty}}}
\newcommand{\oneToInfty}{\ensuremath{\boldsymbol{1^\infty}}}
\newcommand{\zeroToZero}{\ensuremath{\boldsymbol{0^0}}}
\newcommand{\inftyToZero}{\ensuremath{\boldsymbol{\infty^0}}}



\newcommand{\numOverZero}{\ensuremath{\boldsymbol{\tfrac{\#}{0}}}}
\newcommand{\dfn}{\textbf}
%\newcommand{\unit}{\,\mathrm}
\newcommand{\unit}{\mathop{}\!\mathrm}
\newcommand{\eval}[1]{\bigg[ #1 \bigg]}
\newcommand{\seq}[1]{\left( #1 \right)}
\renewcommand{\epsilon}{\varepsilon}
\renewcommand{\phi}{\varphi}


\renewcommand{\iff}{\Leftrightarrow}

\DeclareMathOperator{\arccot}{arccot}
\DeclareMathOperator{\arcsec}{arcsec}
\DeclareMathOperator{\arccsc}{arccsc}
\DeclareMathOperator{\si}{Si}
\DeclareMathOperator{\scal}{scal}
\DeclareMathOperator{\sign}{sign}


%% \newcommand{\tightoverset}[2]{% for arrow vec
%%   \mathop{#2}\limits^{\vbox to -.5ex{\kern-0.75ex\hbox{$#1$}\vss}}}
\newcommand{\arrowvec}[1]{{\overset{\rightharpoonup}{#1}}}
%\renewcommand{\vec}[1]{\arrowvec{\mathbf{#1}}}
\renewcommand{\vec}[1]{{\overset{\boldsymbol{\rightharpoonup}}{\mathbf{#1}}}}
\DeclareMathOperator{\proj}{\mathbf{proj}}
\newcommand{\veci}{{\boldsymbol{\hat{\imath}}}}
\newcommand{\vecj}{{\boldsymbol{\hat{\jmath}}}}
\newcommand{\veck}{{\boldsymbol{\hat{k}}}}
\newcommand{\vecl}{\vec{\boldsymbol{\l}}}
\newcommand{\uvec}[1]{\mathbf{\hat{#1}}}
\newcommand{\utan}{\mathbf{\hat{t}}}
\newcommand{\unormal}{\mathbf{\hat{n}}}
\newcommand{\ubinormal}{\mathbf{\hat{b}}}

\newcommand{\dotp}{\bullet}
\newcommand{\cross}{\boldsymbol\times}
\newcommand{\grad}{\boldsymbol\nabla}
\newcommand{\divergence}{\grad\dotp}
\newcommand{\curl}{\grad\cross}
%\DeclareMathOperator{\divergence}{divergence}
%\DeclareMathOperator{\curl}[1]{\grad\cross #1}
\newcommand{\lto}{\mathop{\longrightarrow\,}\limits}

\renewcommand{\bar}{\overline}

\colorlet{textColor}{black}
\colorlet{background}{white}
\colorlet{penColor}{blue!50!black} % Color of a curve in a plot
\colorlet{penColor2}{red!50!black}% Color of a curve in a plot
\colorlet{penColor3}{red!50!blue} % Color of a curve in a plot
\colorlet{penColor4}{green!50!black} % Color of a curve in a plot
\colorlet{penColor5}{orange!80!black} % Color of a curve in a plot
\colorlet{penColor6}{yellow!70!black} % Color of a curve in a plot
\colorlet{fill1}{penColor!20} % Color of fill in a plot
\colorlet{fill2}{penColor2!20} % Color of fill in a plot
\colorlet{fillp}{fill1} % Color of positive area
\colorlet{filln}{penColor2!20} % Color of negative area
\colorlet{fill3}{penColor3!20} % Fill
\colorlet{fill4}{penColor4!20} % Fill
\colorlet{fill5}{penColor5!20} % Fill
\colorlet{gridColor}{gray!50} % Color of grid in a plot

\newcommand{\surfaceColor}{violet}
\newcommand{\surfaceColorTwo}{redyellow}
\newcommand{\sliceColor}{greenyellow}




\pgfmathdeclarefunction{gauss}{2}{% gives gaussian
  \pgfmathparse{1/(#2*sqrt(2*pi))*exp(-((x-#1)^2)/(2*#2^2))}%
}


%%%%%%%%%%%%%
%% Vectors
%%%%%%%%%%%%%

%% Simple horiz vectors
\renewcommand{\vector}[1]{\left\langle #1\right\rangle}


%% %% Complex Horiz Vectors with angle brackets
%% \makeatletter
%% \renewcommand{\vector}[2][ , ]{\left\langle%
%%   \def\nextitem{\def\nextitem{#1}}%
%%   \@for \el:=#2\do{\nextitem\el}\right\rangle%
%% }
%% \makeatother

%% %% Vertical Vectors
%% \def\vector#1{\begin{bmatrix}\vecListA#1,,\end{bmatrix}}
%% \def\vecListA#1,{\if,#1,\else #1\cr \expandafter \vecListA \fi}

%%%%%%%%%%%%%
%% End of vectors
%%%%%%%%%%%%%

%\newcommand{\fullwidth}{}
%\newcommand{\normalwidth}{}



%% makes a snazzy t-chart for evaluating functions
%\newenvironment{tchart}{\rowcolors{2}{}{background!90!textColor}\array}{\endarray}

%%This is to help with formatting on future title pages.
\newenvironment{sectionOutcomes}{}{}



%% Flowchart stuff
%\tikzstyle{startstop} = [rectangle, rounded corners, minimum width=3cm, minimum height=1cm,text centered, draw=black]
%\tikzstyle{question} = [rectangle, minimum width=3cm, minimum height=1cm, text centered, draw=black]
%\tikzstyle{decision} = [trapezium, trapezium left angle=70, trapezium right angle=110, minimum width=3cm, minimum height=1cm, text centered, draw=black]
%\tikzstyle{question} = [rectangle, rounded corners, minimum width=3cm, minimum height=1cm,text centered, draw=black]
%\tikzstyle{process} = [rectangle, minimum width=3cm, minimum height=1cm, text centered, draw=black]
%\tikzstyle{decision} = [trapezium, trapezium left angle=70, trapezium right angle=110, minimum width=3cm, minimum height=1cm, text centered, draw=black]



\title[Dig-In:]{Taylor polynomials}

\outcome{Compute Taylor polynomials of functions of several variables.} 
\outcome{View the construction of a Taylor polynomial as an iterative process.} 

\begin{document}
\begin{abstract}
  We introduce Taylor polynomials for functions of several variables.
\end{abstract}
\maketitle


Recall the definition of a \textit{Taylor polynomial}:

\begin{definition}
  Let $f:\R\to\R$ be a function whose first $d$ derivatives exist at $x=c$.
  The \dfn{Taylor polynomial} of degree $d$ of $f$ centered at $x=c$ is
  \[
  p_d(x) = \sum_{k=0}^d\frac{f^{(k)}(c)}{k!}(x-c)^k.
  \]
\end{definition}

\begin{question}
  Let $f(x) = \sin(x)$. Compute the degree $7$ Taylor polynomial
  centered at $x=0$.
  \[
  p_7(x)\begin{prompt}
    = \answer{x-x^3/3!+x^5/5!-x^7/7!}
  \end{prompt}
  \]
  \begin{question}
  Let $f(x) = \cos(x)$. Compute the degree $7$ Taylor polynomial
  centered at $x=0$.
  \[
  p_7(x)\begin{prompt}
    = \answer{1-x^2/2+x^4/4!-x^6/6!}
  \end{prompt}
  \]
  \begin{question}
    Let $f(x) = e^x$. Compute the degree $7$ Taylor polynomial
  centered at $x=0$.
    \[
    p_7(x)\begin{prompt}
      = \answer{1 + x + x^2/2+ x^3/3! + x^4/4!+ x^5/5! + x^6/6!+x^7/7!}
    \end{prompt}
    \]
  \end{question}
\end{question}
\end{question}

We have a similar formula for functions $F:\R^n\to \R$:
\begin{definition}
  Let $F:\R^n\to\R$ be a function whose first $d$ derivatives exist at
  $\vec{x}=\vec{c}$.  The \dfn{Taylor polynomial} of degree $d$ of $F$
  centered at $\vec{x}=\vec{c}$ is
  \[
  P_d(\vec{x}) = \sum_{k=0}^d \eval{\eval{\frac{(\vec{a}\dotp \grad)^k F(\vec{x})}{k!}}_{\vec{x}=\vec{c}}}_{\vec{a}=\vec{x}-\vec{c}}
  \]
\end{definition}
Don't panic! This formula is complex and will take some
unpacking. We'll walk you through this.

\subsection{The degree zero Taylor polynomial}
First note that
\begin{align*}
  P_0(\vec{x}) &= \sum_{k=0}^0 \eval{\eval{\frac{(\vec{a}\dotp \grad)^k F(\vec{x})}{k!}}_{\vec{x}=\vec{c}}}_{\vec{a}=\vec{x}-\vec{c}}\\
  &=\eval{\eval{\frac{(\vec{a}\dotp \grad)^0 F(\vec{x})}{0!}}_{\vec{x}=\vec{c}}}_{\vec{a}=\vec{x}-\vec{c}}\\
  &=\eval{\eval{F(\vec{x})}_{\vec{x}=\vec{c}}}_{\vec{a}=\vec{x}-\vec{c}}\\
  &=F(\vec{c}).
\end{align*}
This means for any function $F:\R^n\to\R$, the $0$th degree Taylor
polynomial for $F$ at $\vec{x}=\vec{c}$ is just
\[
P_0(\vec{x})=F(\vec{c}).
\]
\begin{question}
  Consider $F(x,y)= \sin(x+y)$. Compute the $0$th degree Taylor
  polynomial for $F$ at $\vector{x,y} =\vector{\pi/4,\pi/4}$.
  \begin{prompt}
    \[
    P_0(x,y) = \answer{1}
    \]
  \end{prompt}
\end{question}

\subsection{The  degree one Taylor polynomial}
Now let's look at the $1$st degree Taylor polynomial:
\begin{align*}
  P_1(\vec{x})&= \sum_{k=0}^1 \eval{\eval{\frac{(\vec{a}\dotp \grad)^k F(\vec{x})}{k!}}_{\vec{x}=\vec{c}}}_{\vec{a}=\vec{x}-\vec{c}}\\
  &=P_0(\vec{x}) + \eval{\eval{\frac{(\vec{a}\dotp \grad)^1 F(\vec{x})}{1!}}_{\vec{x}=\vec{c}}}_{\vec{a}=\vec{x}-\vec{c}}\\
  &=F(\vec{c}) + \eval{\eval{(\vec{a}\dotp \grad) F(\vec{x})}_{\vec{x}=\vec{c}}}_{\vec{a}=\vec{x}-\vec{c}}
\end{align*}
Now we ask, what is $(\vec{a}\dotp\grad)F$?  Well, computing the dot
product,
\[
(\vec{a}\dotp\grad) = a_1 \pp{x} + a_2 \pp{y}
\]
and to find $(\vec{a}\dotp\grad)F$, we distribute $F$ to obtain
\begin{align*}
  (\vec{a}\dotp\grad)F &= a_1 \pp[F]{x} + a_2 \pp[F]{y}\\
  &= \grad F(\vec{x})\dotp \vec{a}
\end{align*}
So
\begin{align*}
  P_1(\vec{x})&=F(\vec{c}) + \eval{\eval{\grad F(\vec{x})\dotp\vec{a}}_{\vec{x}=\vec{c}}}_{\vec{a}=\vec{x}-\vec{c}}\\
  &=F(\vec{c}) +\grad F(\vec{c})\dotp (\vec{x}-\vec{c}).
\end{align*}
This means for any function $F:\R^n\to\R$, the $1$st degree Taylor
polynomial for $F$ at $\vec{x}=\vec{c}$ is just the tangent ``plane''
for $F$ at $\vec{x}= \vec{c}$.


\begin{question}
  Consider $F(x,y)= 3+4x-5y$. Compute the $1$st degree Taylor
  polynomial for $F$ at $\vector{x,y} =\vector{0,0}$.
  \begin{prompt}
    \[
    P_1(x,y) = \answer{3+4x-5y}
    \]
  \end{prompt}
\end{question}

\subsection{The  degree two Taylor polynomial}
To get our hands on the $2$nd degree Taylor polynomial, we will
specialize to functions $F:\R^2\to\R$. Let $\vec{c}=\vector{c_1,c_2}$
and let $\vec{x} = \vector{x,y}$.  Write with me:
\[
P_2(\vec{x}) = P_1(\vec{x}) + \eval{\eval{\frac{(\vec{a}\dotp \grad)^2 F(\vec{x})}{2!}}_{\vec{x}=\vec{c}}}_{\vec{a}=\vec{x}-\vec{c}}
\]
\begin{align*}
  = F(\vec{c})
  &+ \grad F(\vec{c})\dotp (\vec{x}-\vec{c})\\
  &+(1/2)\eval{\eval{(\vec{a}\dotp \grad)^2 F(\vec{x})}_{\vec{x}=\vec{c}}}_{\vec{a}=\vec{x}-\vec{c}}
\end{align*}
Now we ask ourselves, what is $(\vec{a}\dotp \grad)^2 F(\vec{x})$?
Well, we know that
\[
(\vec{a}\dotp\grad)F = a_1 \pp[F]{x} + a_2 \pp[F]{y}
\]
and 
\begin{align*}
(\vec{a}\dotp\grad)^2F &=(\vec{a}\dotp\grad)(\vec{a}\dotp\grad)F\\
  &= (\vec{a}\dotp\grad)\left(a_1 \pp[F]{x} + a_2 \pp[F]{y}\right)\\
  &= \left(a_1 \pp[F]{x} + a_2 \pp[F]{y}\right)\left(a_1 \pp[F]{x} + a_2 \pp[F]{y}\right)
\end{align*}
Now we use the distributively property and (since we are assuming that
all derivatives of $F$ exist) we have that
\[
\frac{\partial^2F}{\partial x\partial y}  = \frac{\partial^2F}{\partial y\partial x}
\]
so
\[
(\vec{a}\dotp\grad)^2F = a_1^2\frac{\partial^2F}{\partial x^2} + 2a_1
a_2\frac{\partial^2F}{\partial x\partial y} +
a_2^2\frac{\partial^2F}{\partial y^2}.
\]
We can now unpack our second degree Taylor polynomial:
\begin{align*}
  P_2(\vec{x})= F(\vec{c})
&+ \grad F(\vec{c})\dotp (\vec{x}-\vec{c})\\
&+(1/2)\eval{\eval{(\vec{a}\dotp\grad)^2F}_{\vec{x}=\vec{c}}}_{\vec{a}=\vec{x}-\vec{c}},
\end{align*}
as
\begin{align*}
P_2(\vec{x})=F(\vec{c})
&+ \grad F(\vec{c})\dotp (\vec{x}-\vec{c})\\
&+(1/2)\eval{\eval{a_1^2\frac{\partial^2F}{\partial x^2} + 2a_1 
a_2\frac{\partial^2F}{\partial x\partial y} +
a_2^2\frac{\partial^2F}{\partial y^2}}_{\vec{x}=\vec{c}}}_{\vec{a}=\vec{x}-\vec{c}}.
\end{align*}
and finally we see:
\begin{align*}
P_2(\vec{x})=F(\vec{c})
&+ \grad F(\vec{c})\dotp (\vec{x}-\vec{c})\\
&+\left(\frac{1}{2}\right)F^{(2,0)}(\vec{c}) (x-c_1)^2 \\
&+ F^{(1,1)}(\vec{c})(x-c_1)(y-c_2) \\
&+ \left(\frac{1}{2}\right)F^{(0,2)}(\vec{c})(y-c_2)^2.
\end{align*}


\section{Try it, you might like it}

Computing the Taylor polynomial is not so bad, you just need to get the hang of it. 

\begin{example}
  Compute the degree $2$ Taylor polynomial for:
  \[
  F(x,y)=\sin(xy)
  \]
  centered at $(0,0)$.
  \begin{explanation}
    We'll start by making a table of partial derivatives along with
    their value when evaluated at $(0,0)$:
    \[
    \begin{array}{lcl}
      F(x,y) = \sin(xy) & \Rightarrow &F(0,0) = 0\\
      F^{(1,0)}(x,y) = \answer[given]{y\cos(xy)} & \Rightarrow & F^{(1,0)}(0,0) = \answer[given]{0}\\
      F^{(0,1)}(x,y) = \answer[given]{x\cos(xy)} &\Rightarrow  & F^{(0,1)}(0,0) = \answer[given]{0}\\
      F^{(2,0)}(x,y) = \answer[given]{-y^2\sin(xy)} &\Rightarrow & F^{(2,0)}(0,0) = \answer[given]{0}\\
      F^{(0,2)}(x,y) = \answer[given]{-x^2\sin(xy)} &\Rightarrow & F^{(0,2)}(0,0) = \answer[given]{0}\\
      F^{(1,1)}(x,y) = \answer[given]{\cos(xy)-xy\sin(xy)} &\Rightarrow & F^{(1,1)}(0,0) = \answer[given]{1}
    \end{array}
    \]
    We see our polynomial is just
    \begin{align*}
      P_2(x,y) &= \answer[given]{1}\cdot (x-0)\cdot (y-0)\\
      &= \answer[given]{xy}.
    \end{align*}
  \end{explanation}
\end{example}


\begin{example}
  Compute the degree $2$ Taylor polynomial for:
  \[
  F(x,y)= x^2 y + y^2 + x y
  \]
  centered at $(-1,0)$.
  \begin{explanation}
    We'll start by making a table of partial derivatives along with
    their value when evaluated at $(-1,0)$:
    \[
    \begin{array}{lcl}
      F(x,y) = x^2y+y^2+xy & \Rightarrow &F(-1,0) = 0\\
      F^{(1,0)}(x,y) = \answer[given]{2xy+y} & \Rightarrow & F^{(1,0)}(-1,0) = \answer[given]{0}\\
      F^{(0,1)}(x,y) = \answer[given]{x^2+2y+x} &\Rightarrow  & F^{(0,1)}(-1,0) = \answer[given]{0}\\
      F^{(2,0)}(x,y) = \answer[given]{2y} &\Rightarrow & F^{(2,0)}(-1,0) = \answer[given]{0}\\
      F^{(0,2)}(x,y) = \answer[given]{2} &\Rightarrow & F^{(0,2)}(-1,0) = \answer[given]{2}\\
      F^{(1,1)}(x,y) = \answer[given]{1+2x} &\Rightarrow & F^{(1,1)}(-1,0) = \answer[given]{-1}
    \end{array}
    \]
    We see our polynomial is
    \begin{align*}
      P_2(x,y) &= (1/2)\left(\answer[given]{2}\cdot (y-0)^2 + 2 (\answer[given]{-1})(x-(-1))(y-0) \right)\\
      &= \answer[given]{y^2-(x+1)y}.
    \end{align*}
  \end{explanation}
\end{example}


\begin{example}
  Compute the degree $2$ Taylor polynomial for:
  \[
  F(x,y)= x^3-3x-y^2+4y
  \]
  centered at $(-1,2)$.
  \begin{explanation}
    We'll start by making a table of partial derivatives along with
    their value when evaluated at $(-1,2)$:
    \[
    \begin{array}{lcl}
      F(x,y) = x^3-3x-y^2+4y & \Rightarrow &F(-1,2) = 6\\
      F^{(1,0)}(x,y) = \answer[given]{3x^2-3} & \Rightarrow & F^{(1,0)}(-1,2) = \answer[given]{0}\\
      F^{(0,1)}(x,y) = \answer[given]{-2y+4} &\Rightarrow  & F^{(0,1)}(-1,2) = \answer[given]{0}\\
      F^{(2,0)}(x,y) = \answer[given]{6x} &\Rightarrow & F^{(2,0)}(-1,2) = \answer[given]{-6}\\
      F^{(0,2)}(x,y) = \answer[given]{-2} &\Rightarrow & F^{(0,2)}(-1,2) = \answer[given]{-2}\\
      F^{(1,1)}(x,y) = \answer[given]{0} &\Rightarrow & F^{(1,1)}(-1,2) = \answer[given]{0}
    \end{array}
    \]
    We see our polynomial is
    \begin{align*}
      P_2(x,y) &= 6 + (1/2)\left(\answer[given]{-6}\cdot (x-(-1))^2 +  (\answer[given]{-2})(y-2)^2 \right)\\
      &= \answer[given]{6-3(x+1)^2-(y-2)^2}.
    \end{align*}
  \end{explanation}
\end{example}

\begin{question}
  Compute the degree $2$ Taylor polynomial for:
  \[
  F(x,y)=\cos(xy)
  \]
  centered at $(0,0)$.
  \begin{hint}
    Start by making a table of partial derivatives along with
    their value when evaluated at $(0,0)$.
  \end{hint}
    \begin{prompt}
    \[
    P_2(x,y) = \answer{1}
    \]
  \end{prompt}
\end{question}

\begin{question}
  Compute the degree $2$ Taylor polynomial for:
  \[
  F(x,y)= x^2 y + y^2 + x y
  \]
  centered at $(-1/2,1/8)$.
  \begin{hint}
    Start by making a table of partial derivatives along with
    their value when evaluated at $(-1/2,1/8)$.
  \end{hint}
  \begin{prompt}
    \[
    P_2(x,y) = \answer{-1/64 + (1/8)(x+1/2)^2+(y-1/8)^2}
    \]
  \end{prompt}
\end{question}

\begin{question}
  Compute the degree $2$ Taylor polynomial for:
  \[
  F(x,y)= x^3-3x-y^2+4y
  \]
  centered at $(1,2)$.
  \begin{hint}
    Start by making a table of partial derivatives along with
    their value when evaluated at $(1,2)$.
  \end{hint}
  \begin{prompt}
    \[
    P_2(x,y) = \answer{2 + 3(x-1)^2 -(y-2)^2}
    \]
  \end{prompt}
\end{question}

\section{In other words}

Now that we have a formula and we (hopefully!) can apply it. Let's
finish by talking about what is really going on. Given a function
$f:\R\to\R$, the Taylor polynomial 
\[
p_d(x) = \sum_{k=0}^d\frac{f^{(k)}(c)}{k!}(x-c)^k.
\]
is a polynomial ``cooked-up'' to share the value of the function,
meaning
\[
p_d(c)=f(c),
\]
and share values of the first $d$ derivatives, meaning
\[
p_d^{(i)}(c) = f^{(i)}(c) 
\]
whenever $0\le i\le d$. The exact same idea is true for functions of
several variables. Let's explain the construction of the Taylor polynomial as an iterative process. Given $F:\R^2\to\R$ (and similarly for functions
$F:\R^n\to\R$) the degree zero Taylor polynomial is just the value of
the function
\[
P_0(x,y) = F(c_1,c_2)
\]
where $\vec{c}=\vector{c_1,c_2}$ is the center of the Taylor
polynomial.  The degree one Taylor polynomial is just the degree zero
polynomial plus the first partial derivatives with respect to $x_i$
multiplied by $(x_i-c_i)$
\[
P_1(x,y) = P_0(x,y) + F^{(1,0)}(\vec{c})(x-c_1) + F^{(0,1)}(\vec{c})(y-c_2).
\]
The degree two Taylor polynomial can be found by adding the degree one
Taylor polynomial to one-half of all the second partial derivatives
with respect to $x$ and $y$ multiplied by
\begin{itemize}
\item $(x-c_1)^2$ when taking the partial derivative with respect to
  $x$,
\item $(y-c_2)^2$ when taking the partial derivative with respect to
  $y$, and
\item $2(x-c_1)(y-c_2)$ when taking the partial derivative with respect
  to $x$ and $y$.
\end{itemize}
Putting this together, we see
\begin{align*}
P_2(x,y) &= P_1(x,y) \\
&+\left(\frac{1}{2}\right)F^{(2,0)}(\vec{c})(x-c_1)^2 \\
&+F^{(1,1)}(\vec{c})(x-c_1)(y-c_2)\\
&+\left(\frac{1}{2}\right)F^{(0,2)}(\vec{c})(y-c_2)^2.
\end{align*}
The interested reader can (repeatedly) differentiate $P_2(x,y)$ to see
that its value at $\vec{x}=\vec{c}$ and the values of the first two
derivatives of $P_2(x,y)$ do indeed match those of $F(x,y)$.

\end{document}
