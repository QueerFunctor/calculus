\documentclass[]{ximera}
%handout:  for handout version with no solutions or instructor notes
%handout,instructornotes:  for instructor version with just problems and notes, no solutions
%noinstructornotes:  shows only problem and solutions

%% handout
%% space
%% newpage
%% numbers
%% nooutcomes

%I added the commands here so that I would't have to keep looking them up
%\newcommand{\RR}{\mathbb R}
%\renewcommand{\d}{\,d}
%\newcommand{\dd}[2][]{\frac{d #1}{d #2}}
%\renewcommand{\l}{\ell}
%\newcommand{\ddx}{\frac{d}{dx}}
%\everymath{\displaystyle}
%\newcommand{\dfn}{\textbf}
%\newcommand{\eval}[1]{\bigg[ #1 \bigg]}

%\begin{image}
%\includegraphics[trim= 170 420 250 180]{Figure1.pdf}
%\end{image}

%add a ``.'' below when used in a specific directory.

%\usepackage{todonotes}

\newcommand{\todo}{}

\usepackage{esint} % for \oiint
\ifxake%%https://math.meta.stackexchange.com/questions/9973/how-do-you-render-a-closed-surface-double-integral
\renewcommand{\oiint}{{\large\bigcirc}\kern-1.56em\iint}
\fi


\graphicspath{
  {./}
  {ximeraTutorial/}
  {basicPhilosophy/}
  {functionsOfSeveralVariables/}
  {normalVectors/}
  {lagrangeMultipliers/}
  {vectorFields/}
  {greensTheorem/}
  {shapeOfThingsToCome/}
  {dotProducts/}
  {partialDerivativesAndTheGradientVector/}
  {../productAndQuotientRules/exercises/}
  {../normalVectors/exercisesParametricPlots/}
  {../continuityOfFunctionsOfSeveralVariables/exercises/}
  {../partialDerivativesAndTheGradientVector/exercises/}
  {../directionalDerivativeAndChainRule/exercises/}
  {../commonCoordinates/exercisesCylindricalCoordinates/}
  {../commonCoordinates/exercisesSphericalCoordinates/}
  {../greensTheorem/exercisesCurlAndLineIntegrals/}
  {../greensTheorem/exercisesDivergenceAndLineIntegrals/}
  {../shapeOfThingsToCome/exercisesDivergenceTheorem/}
  {../greensTheorem/}
  {../shapeOfThingsToCome/}
  {../separableDifferentialEquations/exercises/}
}

\newcommand{\mooculus}{\textsf{\textbf{MOOC}\textnormal{\textsf{ULUS}}}}

\usepackage{tkz-euclide}\usepackage{tikz}
\usepackage{tikz-cd}
\usetikzlibrary{arrows}
\tikzset{>=stealth,commutative diagrams/.cd,
  arrow style=tikz,diagrams={>=stealth}} %% cool arrow head
\tikzset{shorten <>/.style={ shorten >=#1, shorten <=#1 } } %% allows shorter vectors

\usetikzlibrary{backgrounds} %% for boxes around graphs
\usetikzlibrary{shapes,positioning}  %% Clouds and stars
\usetikzlibrary{matrix} %% for matrix
\usepgfplotslibrary{polar} %% for polar plots
\usepgfplotslibrary{fillbetween} %% to shade area between curves in TikZ
\usetkzobj{all}
\usepackage[makeroom]{cancel} %% for strike outs
%\usepackage{mathtools} %% for pretty underbrace % Breaks Ximera
%\usepackage{multicol}
\usepackage{pgffor} %% required for integral for loops



%% http://tex.stackexchange.com/questions/66490/drawing-a-tikz-arc-specifying-the-center
%% Draws beach ball
\tikzset{pics/carc/.style args={#1:#2:#3}{code={\draw[pic actions] (#1:#3) arc(#1:#2:#3);}}}



\usepackage{array}
\setlength{\extrarowheight}{+.1cm}
\newdimen\digitwidth
\settowidth\digitwidth{9}
\def\divrule#1#2{
\noalign{\moveright#1\digitwidth
\vbox{\hrule width#2\digitwidth}}}





\newcommand{\RR}{\mathbb R}
\newcommand{\R}{\mathbb R}
\newcommand{\N}{\mathbb N}
\newcommand{\Z}{\mathbb Z}

\newcommand{\sagemath}{\textsf{SageMath}}


%\renewcommand{\d}{\,d\!}
\renewcommand{\d}{\mathop{}\!d}
\newcommand{\dd}[2][]{\frac{\d #1}{\d #2}}
\newcommand{\pp}[2][]{\frac{\partial #1}{\partial #2}}
\renewcommand{\l}{\ell}
\newcommand{\ddx}{\frac{d}{\d x}}

\newcommand{\zeroOverZero}{\ensuremath{\boldsymbol{\tfrac{0}{0}}}}
\newcommand{\inftyOverInfty}{\ensuremath{\boldsymbol{\tfrac{\infty}{\infty}}}}
\newcommand{\zeroOverInfty}{\ensuremath{\boldsymbol{\tfrac{0}{\infty}}}}
\newcommand{\zeroTimesInfty}{\ensuremath{\small\boldsymbol{0\cdot \infty}}}
\newcommand{\inftyMinusInfty}{\ensuremath{\small\boldsymbol{\infty - \infty}}}
\newcommand{\oneToInfty}{\ensuremath{\boldsymbol{1^\infty}}}
\newcommand{\zeroToZero}{\ensuremath{\boldsymbol{0^0}}}
\newcommand{\inftyToZero}{\ensuremath{\boldsymbol{\infty^0}}}



\newcommand{\numOverZero}{\ensuremath{\boldsymbol{\tfrac{\#}{0}}}}
\newcommand{\dfn}{\textbf}
%\newcommand{\unit}{\,\mathrm}
\newcommand{\unit}{\mathop{}\!\mathrm}
\newcommand{\eval}[1]{\bigg[ #1 \bigg]}
\newcommand{\seq}[1]{\left( #1 \right)}
\renewcommand{\epsilon}{\varepsilon}
\renewcommand{\phi}{\varphi}


\renewcommand{\iff}{\Leftrightarrow}

\DeclareMathOperator{\arccot}{arccot}
\DeclareMathOperator{\arcsec}{arcsec}
\DeclareMathOperator{\arccsc}{arccsc}
\DeclareMathOperator{\si}{Si}
\DeclareMathOperator{\scal}{scal}
\DeclareMathOperator{\sign}{sign}


%% \newcommand{\tightoverset}[2]{% for arrow vec
%%   \mathop{#2}\limits^{\vbox to -.5ex{\kern-0.75ex\hbox{$#1$}\vss}}}
\newcommand{\arrowvec}[1]{{\overset{\rightharpoonup}{#1}}}
%\renewcommand{\vec}[1]{\arrowvec{\mathbf{#1}}}
\renewcommand{\vec}[1]{{\overset{\boldsymbol{\rightharpoonup}}{\mathbf{#1}}}}
\DeclareMathOperator{\proj}{\mathbf{proj}}
\newcommand{\veci}{{\boldsymbol{\hat{\imath}}}}
\newcommand{\vecj}{{\boldsymbol{\hat{\jmath}}}}
\newcommand{\veck}{{\boldsymbol{\hat{k}}}}
\newcommand{\vecl}{\vec{\boldsymbol{\l}}}
\newcommand{\uvec}[1]{\mathbf{\hat{#1}}}
\newcommand{\utan}{\mathbf{\hat{t}}}
\newcommand{\unormal}{\mathbf{\hat{n}}}
\newcommand{\ubinormal}{\mathbf{\hat{b}}}

\newcommand{\dotp}{\bullet}
\newcommand{\cross}{\boldsymbol\times}
\newcommand{\grad}{\boldsymbol\nabla}
\newcommand{\divergence}{\grad\dotp}
\newcommand{\curl}{\grad\cross}
%\DeclareMathOperator{\divergence}{divergence}
%\DeclareMathOperator{\curl}[1]{\grad\cross #1}
\newcommand{\lto}{\mathop{\longrightarrow\,}\limits}

\renewcommand{\bar}{\overline}

\colorlet{textColor}{black}
\colorlet{background}{white}
\colorlet{penColor}{blue!50!black} % Color of a curve in a plot
\colorlet{penColor2}{red!50!black}% Color of a curve in a plot
\colorlet{penColor3}{red!50!blue} % Color of a curve in a plot
\colorlet{penColor4}{green!50!black} % Color of a curve in a plot
\colorlet{penColor5}{orange!80!black} % Color of a curve in a plot
\colorlet{penColor6}{yellow!70!black} % Color of a curve in a plot
\colorlet{fill1}{penColor!20} % Color of fill in a plot
\colorlet{fill2}{penColor2!20} % Color of fill in a plot
\colorlet{fillp}{fill1} % Color of positive area
\colorlet{filln}{penColor2!20} % Color of negative area
\colorlet{fill3}{penColor3!20} % Fill
\colorlet{fill4}{penColor4!20} % Fill
\colorlet{fill5}{penColor5!20} % Fill
\colorlet{gridColor}{gray!50} % Color of grid in a plot

\newcommand{\surfaceColor}{violet}
\newcommand{\surfaceColorTwo}{redyellow}
\newcommand{\sliceColor}{greenyellow}




\pgfmathdeclarefunction{gauss}{2}{% gives gaussian
  \pgfmathparse{1/(#2*sqrt(2*pi))*exp(-((x-#1)^2)/(2*#2^2))}%
}


%%%%%%%%%%%%%
%% Vectors
%%%%%%%%%%%%%

%% Simple horiz vectors
\renewcommand{\vector}[1]{\left\langle #1\right\rangle}


%% %% Complex Horiz Vectors with angle brackets
%% \makeatletter
%% \renewcommand{\vector}[2][ , ]{\left\langle%
%%   \def\nextitem{\def\nextitem{#1}}%
%%   \@for \el:=#2\do{\nextitem\el}\right\rangle%
%% }
%% \makeatother

%% %% Vertical Vectors
%% \def\vector#1{\begin{bmatrix}\vecListA#1,,\end{bmatrix}}
%% \def\vecListA#1,{\if,#1,\else #1\cr \expandafter \vecListA \fi}

%%%%%%%%%%%%%
%% End of vectors
%%%%%%%%%%%%%

%\newcommand{\fullwidth}{}
%\newcommand{\normalwidth}{}



%% makes a snazzy t-chart for evaluating functions
%\newenvironment{tchart}{\rowcolors{2}{}{background!90!textColor}\array}{\endarray}

%%This is to help with formatting on future title pages.
\newenvironment{sectionOutcomes}{}{}



%% Flowchart stuff
%\tikzstyle{startstop} = [rectangle, rounded corners, minimum width=3cm, minimum height=1cm,text centered, draw=black]
%\tikzstyle{question} = [rectangle, minimum width=3cm, minimum height=1cm, text centered, draw=black]
%\tikzstyle{decision} = [trapezium, trapezium left angle=70, trapezium right angle=110, minimum width=3cm, minimum height=1cm, text centered, draw=black]
%\tikzstyle{question} = [rectangle, rounded corners, minimum width=3cm, minimum height=1cm,text centered, draw=black]
%\tikzstyle{process} = [rectangle, minimum width=3cm, minimum height=1cm, text centered, draw=black]
%\tikzstyle{decision} = [trapezium, trapezium left angle=70, trapezium right angle=110, minimum width=3cm, minimum height=1cm, text centered, draw=black]




\author{Tom Needham}

\outcome{Understand notation for sequences.}
\outcome{Manipulate and combine sequences defined recursively or by explicit formulas.} 

\title[]{Introduction to Sequences}

\begin{document}
\begin{abstract}
\end{abstract}
\maketitle

\vspace{-0.4in}

\section{Discussion Questions}

\begin{problem}
Classify the sequences defined by the following formulas as geometric, arithmetic or neither.
\begin{center}
\begin{tabular}{lll}
I. $a_n = 5n + 7$ \hspace{.4in} II. $b_n = n^2 + 1$ \hspace{.4in} III. $c_1 = 10, \; c_{n+1}= 2\cdot c_n$
\end{tabular}
\end{center}
\end{problem}

\begin{freeResponse}
I. The sequence $\{a_n\}$ is arithmetic, since 
$$
a_{n+1}-a_n = 5(n+1)+7-(5n+7) = 5n + 5 + 7 - 5n - 7 = 5
$$
is a constant for all $n$.

II. The sequence $\{b_n\}$ is not arithmetic, since
$$
b_{n+1}-b_n = (n+1)^2 + 1 - (n^2+1) = n^2 + 2n + 1 + 1 - n^2 - 1 = 2n + 2
$$
is not constant. It is also not geometric, since
$$
\frac{b_{n+1}}{b_n} = \frac{(n+1)^2 + 1}{n^2 + 1} = \frac{n^2+2n + 2}{n^2 + 1}
$$
is not constant.

III. The sequence $\{c_n\}$ is geometric. This is true  because, for $n\geq 1$, we have
$$
\frac{c_{n+1}}{c_n} = \frac{2 c_n}{c_n} = 2,
$$
which is constant.
\end{freeResponse}


\section{Group Work}

\begin{problem}
Let $\{a_n\}_{n=0}^\infty$ be the sequence defined by the recursive formula
$$
a_0 = 1, \; a_{n+1}=\frac{n+1}{2} \cdot a_n.
$$
Write out the first four terms of the sequence.

Let $\{b_n\}_{n=0}^\infty$ be the sequence defined by the explicit formula 
$$
b_n = \frac{n!}{2^n}.
$$
Write out the first four terms of the sequence.

You should notice that the first four terms of the sequences are the same. Show that for \emph{every} $n \geq 0$, $a_n = b_n$. 
\end{problem}

\begin{freeResponse}
The first four terms of $\{a_n\}$ are
\begin{align*}
a_0 &= 1 \\
a_1 &= \frac{1}{2} a_0 = \frac{1}{2} \\
a_2 &= \frac{2}{2} a_1 = \frac{1}{2} \\
a_3 &= \frac{3}{2} a_2 = \frac{3}{4}.
\end{align*}
The first four terms of $\{b_n\}$ are 
\begin{align*}
b_0 &= \frac{0!}{2^0} = 1\\
b_1 &= \frac{1!}{2^1} = \frac{1}{2}\\
b_2 &= \frac{2!}{2^2} = \frac{2}{4} = \frac{1}{2} \\
b_3 &= \frac{3!}{2^3} = \frac{6}{8} = \frac{3}{4}.\\
\end{align*}

To show that $a_n = b_n$ for all $n \geq 0$, we will show that the terms in $\{b_n\}$ satisfy the recursion formula defining $\{a_n\}$. We already have shown that $b_0 = 1$. Moreover, we have
$$
b_{n+1} = \frac{(n+1)!}{2^{n+1}} = \frac{(n+1) \cdot n!}{2 \cdot 2^n} = \frac{n+1}{2} \cdot \frac{n!}{2^n} = \frac{n+1}{2} b_n.
$$
\end{freeResponse}

\pagebreak

\begin{problem}
Let $\{a_n\}_{n=1}^\infty$ be the sequence given by the formula 
$$
a_n = \frac{1}{n!}
$$
Write the out the first four terms of each of sequences defined by the following formulas. Each sequence is defined for $n \geq 1$. 
\begin{center}
\begin{tabular}{lll}
I. $s_n = \sum_{j=1}^n a_j$ \hspace{.4in} II. $b_n = \frac{a_{n+1}}{a_n}$ \hspace{.4in} III. $c_n = \sqrt[n]{a_n}$
\end{tabular}
\end{center}
For the sequence defined in part II, find a simplified form for the $n$th term $b_n$. 
\end{problem}

\begin{freeResponse}
I. The first four terms of $\{s_n\}$ are 
\begin{align*}
s_1 &= a_1 = 1\\
s_2 &= a_1 + a_2 = 1 + \frac{1}{2} = \frac{3}{2}\\
s_3 &= s_2 + a_3 = \frac{3}{2} + \frac{1}{6} = \frac{5}{3} \\
s_4 &= s_3 + a_4  = \frac{5}{3} + \frac{1}{24} = \frac{41}{24}.\\
\end{align*}

II. The first four terms of $\{b_n\}$ are 
\begin{align*}
b_1 &= \frac{a_2}{a_1} = \frac{1/2}{1} = \frac{1}{2} \\
b_2 &= \frac{a_3}{a_2} = \frac{1/6}{1/2} = \frac{1}{3} \\
b_3 &= \frac{a_4}{a_3} = \frac{1/24}{1/6} = \frac{1}{4} \\
b_4 &= \frac{a_5}{a_4} = \frac{1/120}{1/24} = \frac{1}{5}.\\
\end{align*}
After looking at these terms, one might guess that $b_n = \frac{1}{n+1}$ for all $n \geq 1$. Indeed, we have
$$
b_n = \frac{a_{n+1}}{a_n} = \frac{1/(n+1)!}{1/n!} = \frac{n!}{(n+1)!} = \frac{n!}{(n+1) \cdot n!} = \frac{1}{n+1}.
$$

III. The first four terms of $\{c_n\}$ are
\begin{align*}
c_1 &= \sqrt[1]{a_1} = 1\\
c_2 &= \sqrt[2]{a_2} = \sqrt{\frac{1}{2}} \\
c_3 &= \sqrt[3]{a_3} = \sqrt[3]{\frac{1}{6}} \\
c_4 &= \sqrt[4]{a_4} = \sqrt[4]{\frac{1}{24}}.\\
\end{align*}
\end{freeResponse}

\begin{problem}
Let $\{a_n\}_{n=1}^\infty$ be the sequence defined by the formula
$$
a_n = \frac{n+2}{3^n}
$$
and let $\{b_n\}_{n=0}^\infty$ be the sequence defined recursively by
$$
b_0 = 1 \mbox{ and } b_{n+1} = b_n + \frac{1}{b_n}.
$$
Write the out the first four terms of each of sequences defined by the following formulas.
\begin{center}
\begin{tabular}{lll}
I. $\{a_n+b_n\}_{n=1}^\infty$ \hspace{.3in} II. $\{a_{n+1}/b_n\}_{n=0}^\infty$ \hspace{.3in} III. $c_1 = 1, \; c_{n+1} = c_n + a_n + b_n$
\end{tabular}
\end{center}
\end{problem}

\begin{freeResponse}
It seems that it would be useful to first write out several terms of the squences $\{a_n\}$ and $\{b_n\}$:
\begin{align*}
a_1 &= \frac{3}{3} = 1 \\
a_2 &= \frac{4}{3^2} = \frac{4}{9} \\
a_3 &= \frac{5}{3^3} = \frac{5}{27} \\
a_4 &= \frac{6}{3^4} = \frac{6}{81} = \frac{2}{27}\\
\end{align*}
and 
\begin{align*}
b_0 &= 1\\
b_1 &= 1 + \frac{1}{1} = 2\\
b_2 &= 2 + \frac{1}{2} = \frac{5}{2}\\
b_3 &= \frac{5}{2} + \frac{2}{5} = \frac{29}{10} \\
b_4 &= \frac{29}{10} + \frac{10}{29} = \frac{29^2 + 10^2}{290} = \frac{941}{290}\\
\end{align*}

I. The first four terms of the sequence $\{a_n + b_n\}_{n=1}^\infty$ are 
\begin{align*}
a_1 + b_1 &= 1 + 2 = 3\\
a_2 + b_2 &= \frac{4}{9} + \frac{5}{2}\\
a_3 + b_3 &= \frac{5}{27} + \frac{29}{10} \\
a_4 + b_4 &= \frac{2}{27} + \frac{941}{290}.\\
\end{align*}

II. The first four terms of the sequence $\{a_{n+1}/b_n\}$ are 
\begin{align*}
\frac{a_1}{b_0} &= 1 \\
\frac{a_2}{b_1} &= \frac{4/9}{2} = \frac{2}{9} \\
\frac{a_3}{b_2} &= \frac{5/27}{5/2} = \frac{2}{27}\\
\frac{a_4}{b_3} &= \frac{2/27}{29/10} = \frac{20}{783}.\\
\end{align*}

III. The first four terms of $\{c_n\}$ are 
\begin{align*}
c_1 &= 1 \\
c_2 &= c_1 + a_1 + b_1 = 1 + 3 = 4 \\
c_3 &= c_2 + a_2 + b_2 = 4 + \frac{4}{9} + \frac{5}{2} \\
c_4 &= c_3 + a_3 + b_3 =  4 + \frac{4}{9} + \frac{5}{2} + \frac{5}{27} + \frac{29}{10}\\
\end{align*}
\end{freeResponse}


\end{document}
